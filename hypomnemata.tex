\documentclass[11pt]{article}
\usepackage{fullpage}
\usepackage{fontspec}
\usepackage{xunicode}
\usepackage{xltxtra}
\defaultfontfeatures{Mapping=tex-text}
%\setromanfont{Linux Libertine O}
\setromanfont{Brill}
%\newfontfamily\greek{Gentium}


\usepackage{multicol}
\usepackage{ulem}
\usepackage{ifthen}
% Interdocument linking.
\usepackage[xetex]{hyperref}
\hypersetup{bookmarksopen=false,
  pdfpagemode=UseNone,
  colorlinks=true,
  urlcolor=blue,
  linkcolor=black,    % no links for footnotes; URLs will still have color
  pdftitle={Glossopoetic Hypmnemata},
  pdfauthor={William S. Annis},
  pdfkeywords={conlang}%,
}


% Better sectioning for this document.
\usepackage[compact,rigidchapters,explicit]{titlesec}
\setcounter{secnumdepth}{4}
\titleformat{\section}[display]
 {\normalfont\fillast}
 {\normalfont\bfseries \thesection. #1}
 {1ex minus .1ex}
 {\small}
\titlespacing{\section}{3pc}{*4}{-1em}[3pc]

\titleformat{\subsection}[runin]{\normalfont\bfseries}{\thesubsection.}{.5em}{#1. }[ { }]
\titlespacing{\subsection}{1ex}{1.5ex plus .1ex minus .2ex}{0pt}

\titleformat{\subsubsection}[runin]{\normalfont\bfseries\small}{\thesubsubsection.}{.5em}{#1. }[ { }]
\titlespacing{\subsubsection}{1ex}{1.5ex plus .1ex minus .2ex}{0pt}

% If no argument is given, only the section is printed, no title.
\titleformat{\paragraph}[runin]{\normalfont\bfseries\small}{\theparagraph.}{.5em}{\ifthenelse{\equal{#1}{}}{}{#1. }}[ { }]
\titlespacing{\paragraph}{1ex}{1.5ex plus .1ex minus .2ex}{0pt}

% Some utilities.
\newcommand{\LL}[1]{\textbf{#1}}  % Other language
\newcommand{\E}[1]{\textit{#1}}   % English
\newcommand{\I}[1]{\textsc{#1}}   % Interlinears
\newcommand{\note}[1]{\textcolor{magenta}{\small\textit{#1}}}
\newcommand{\tsref}[1]{\hyperref[#1]{\S \textbf{\ref*{#1}}}}
\newcommand{\interlin}[1]{\begin{quotation}{\small\noindent#1}\end{quotation}}
\newenvironment{grammarlist}%
 {\begin{itemize}\addtolength{\itemsep}{-0.5\baselineskip}\ignorespaces}%
 {\end{itemize}\ignorespacesafterend}

\newenvironment{dlist}%
 {\begin{quote}\begin{description}\addtolength{\itemsep}{-0.3\baselineskip}\ignorespaces}%
 {\end{description}\ignorespacesafterend\end{quote}}

\newenvironment{examples}{\quote}{\endquote}
\newcommand{\example}[2]{\noindent\LL{#1}\hskip1em\E{#2}}
\newcommand{\longexample}[2]{\noindent\LL{#1}
\indent\E{#2}}

\begin{document}
\frenchspacing
\title{Glossopoetic Hypomnemata}
\author{Wm S. Annis}
\date{\today}
\maketitle

\section{Phonology and Phonemes}

Illegal clusters may occur in particular grammatical contexts, and
thus look common (\textit{cf.} Latin \LL{-nt} in \I{3pl} verb
endings).


\section{The Noun}

Nouns are the most frequently borrowed word class.


\section{The Pronoun}

Languages with 1/2 systems are not common, but not rare either.

Sg/du/paucal/pl is far more common that sg/du/trial/pl.

It's not especially common for sg and pl forms to be related (as in
Chinese).

Many languages:

\begin{center}
\begin{tabular}{lll}
  & Singular & Plural \\
\I{1st} & 1 & 12, 13 (and 11) \\
\I{2nd} & 2 & 22, 23 \\
\I{3rd} & 3 & 33
\end{tabular}
\end{center}

However, one might get something like:

\begin{center}
\begin{tabular}{ll}
1 & 13 (and 11) \\
2 & 12, 22, 23     
\end{tabular}
\end{center}

The basic 1/2/3 sg/pl system may be extended to include a special 12
``me and you'' form.  This may involve the innovation of a dual
throughout the pronoun system.  Or it may be just a normal system with
an inclusive/exclusive distinction.  Or a minimal/augmented system, 

\begin{center}
\begin{tabular}{ll}
Minimal & Non-minimal \\
1  & 1 + others \\
1+2 & 1+2 + others \\
2 & 2 + others \\
3 & 3 + others 
\end{tabular}
\end{center}

\noindent The non-minimal may include ``unit augmented'' (one other
person, producing a dual in many forms) and just ``augmented'' (the
``plural'').  Minimal/augmented are common in Australia, Austronesian,
South America, and a few in North America, though the unit
augmented/augmented distinction is nearly restricted to Australia.

3sg = 3pl is a common neutralization.  2pl = 3pl (and 2du = 3du) can
happen.

Using 2du, 2paucal or 2pl as a mark of respect occurs in Europe,
Oceania, Australia.

There is a relationship between the indefinite ``someone'' and
1non-sg forms (French; Caddo, etc).  Usually it's 1pl.inc that does
this. 


\section{The Adjective}


\section{The Verb}

``Labile transitivity'' is very common.  Be clear on \I{s = a}
\textit{vs.} \I{s = o} for intransitive constructions.  Some languages
may skew in favor of one or the other, but others do not.  English
``cook,'' with transitive and both intransitives (``the chef cooks;
the chef cooks stew; the stew is cooking'') is unusual.

A transitive verb may be used intransitively in an extended
intransitive sense, too.

Among other categories, some languages have verbs that mark ease or
difficulty.  The marking for ``ease'' on a transitive verb may signal
a small \I{o}.

\subsection{Semantic Types and Their Frames}

\begin{center}
\begin{tabular}{lllll}
\I{affect} (\E{hit, cut}) & Agent & Target & Instrument \\
\I{giving} (\E{give, lend}) & Donor & Gift & Recipient \\
\I{speaking} (\E{speak, tell}) & Speaker & Addressee & Message & Medium \\
\I{thinking} (\E{consider}) & Cogitator & Thought \\
\I{attention} (\E{see, hear}) & Perceiver & Impression \\
\I{liking} (\E{like, love, hate}) & Experiencer & Stimulus
\end{tabular}
\end{center}

In some languages, an \E{affect} verb may require an inanimate agent,
usually specific, such as food making a person sick, \textit{etc}.

In very few languages are \I{liking} verbs like \I{annoying} verbs,
where the experiencer and stimulus are in \I{o} and \I{a} roles.
However, extended intransitives, with an oblique experiencer, are a
bit more common.

\I{attention} and \I{liking} may have extended intransitive frames.
Or dative subjects.  Or oblique objects.

The types with more than two arguments may have lexical or
construction splits to determine which of the arguments is in the
primary \I{o} slot (``tell'' vs. ``speak (French)'' vs. ``say'').

For \I{give} there may be i) single verb with different constructions,
ii) gift role is always in \I{o} function, iii) recipient role always
in \I{o} function (rare; amusingly, with \I{dat} for the gift role).
Or, ii and iii may have different lexical items.


\section{Number}

The distinction between ``paucal'' and ``plural'' is context
dependent. 

\section{Constructions}

Any schematic construction may, like lexical constructions, be
polysemous.  For example, the English ditransitive:

\begin{grammarlist}
  \item \I{X causes Y to receive Z} (central sense), ``Joe gave Sally
    the ball.''
  \item Conditions of satisfaction imply \I{X causes Y to receive Z},
    ``Joe promised Bob a car.''
  \item \I{X causes Y not to receive Z}, ``Joe refused Bob a cookie.''
  \item \I{X acts to cause Y to receive Z} at some future time, ``Joe
    bequeathed Bob a fortune.''
  \item \I{X enables Y to receive Z}, ``Joe permitted Chris an
    apple.''
  \item \I{X intends to cause Y to receive Z}, ``Joe baked Bob a cake.''
\end{grammarlist}

Similarly, a particular language may have one construction for a
particular job, such as a ditransitive construction, but others have
several with different pragmatic or pivot significance (``I gave him
the book'' \textit{vs.} ``I gave the book to him'').

The non-ergative clauses in so-called split ergative systems are
rarely nom-acc (\I{a = nom, p = acc, s = nom}).  Instead one can get
\I{a = abs, p = abs, s = abs} (common), \I{a = abs, p = obl, s = abs}
(common) or even \I{a = erg, p = abs, s = erg} (common in Mayan).  The
aspect splits reflect (historically) intransitive \I{aux}
constructions.  Thus, in rare circumstances, one may get splits in
future clauses, or with negation, if they reflected an original
\I{aux} construction.

\end{document}
