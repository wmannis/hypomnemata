\documentclass[11pt]{article}
\usepackage{amssymb} % has to be used before XeTeX unicode trickery.
\usepackage{fullpage}
\usepackage{fontspec}
\usepackage{xunicode}
\usepackage{xltxtra}
\defaultfontfeatures{Mapping=tex-text}
%\setromanfont{Linux Libertine O}
\setromanfont{Brill}
%\newfontfamily\greek{Gentium}


\usepackage{multicol}
\usepackage{ulem}
\usepackage{ifthen}
% Interdocument linking.
\usepackage[xetex]{hyperref}
\hypersetup{bookmarksopen=false,
  pdfpagemode=UseNone,
  colorlinks=true,
  urlcolor=blue,
  linkcolor=black,    % no links for footnotes; URLs will still have color
  pdftitle={Hypmnemata Glossopoetica},
  pdfauthor={William S. Annis},
  pdfkeywords={conlang}%,
}


% Better sectioning for this document.
\usepackage[compact,rigidchapters,explicit]{titlesec}
\setcounter{secnumdepth}{4}
\titleformat{\section}[display]
 {\normalfont\fillast}
 {\normalfont\bfseries \thesection. #1}
 {1ex minus .1ex}
 {\small}
\titlespacing{\section}{3pc}{*4}{-1em}[3pc]

\titleformat{\subsection}[runin]{\normalfont\bfseries}{\thesubsection.}{.5em}{#1. }[ { }]
\titlespacing{\subsection}{1ex}{1.5ex plus .1ex minus .2ex}{0pt}

\titleformat{\subsubsection}[runin]{\normalfont\bfseries\small}{\thesubsubsection.}{.5em}{#1. }[ { }]
\titlespacing{\subsubsection}{1ex}{1.5ex plus .1ex minus .2ex}{0pt}

% If no argument is given, only the section is printed, no title.
\titleformat{\paragraph}[runin]{\normalfont\bfseries\small}{\theparagraph.}{.5em}{\ifthenelse{\equal{#1}{}}{}{#1. }}[ { }]
\titlespacing{\paragraph}{1ex}{1.5ex plus .1ex minus .2ex}{0pt}

% Some utilities.
\newcommand{\LL}[1]{\textbf{#1}}  % Other language
\newcommand{\E}[1]{\textit{#1}}   % English
\newcommand{\I}[1]{\textsc{#1}}   % Interlinears
\newcommand{\note}[1]{\textcolor{magenta}{\small\textit{#1}}}
\newcommand{\tsref}[1]{\hyperref[#1]{\S \textbf{\ref*{#1}}}}
\newcommand{\interlin}[1]{\begin{quotation}{\small\noindent#1}\end{quotation}}
\newcommand{\rara}[1]{$\mathfrak{R}$: #1}
\newenvironment{grammarlist}%
 {\begin{itemize}\addtolength{\itemsep}{-0.5\baselineskip}\ignorespaces}%
 {\end{itemize}\ignorespacesafterend}

\newenvironment{dlist}%
 {\begin{quote}\begin{description}\addtolength{\itemsep}{-0.3\baselineskip}\ignorespaces}%
 {\end{description}\ignorespacesafterend\end{quote}}

\newenvironment{examples}{\quote}{\endquote}
\newcommand{\example}[2]{\noindent\LL{#1}\hskip1em\E{#2}}
\newcommand{\longexample}[2]{\noindent\LL{#1}
\indent\E{#2}}

\begin{document}
\frenchspacing
\title{Hypomnemata Glossopoetica}
\author{Wm S. Annis}
\date{\today}
\maketitle

\section{Phonology}

Illegal clusters may occur in particular grammatical contexts, and
thus look common (\textit{cf.} Latin \LL{-nt} in \I{3pl} verb
endings).

% http://www.unish.org/upload/word/사본%20-%20MarkVanDam%5B1%5D.pdf
Hierarchy of codas:\footnote{Some attested single-\I{c} coda
  inventories: \{n ŋ\}, \{n ŋ t k\}, \{n ŋ m p t k\}, \{n m l r\}, \{n
  m w j\}, \{n ŋ m l r j\}, \{n m r d\}, \{d l s x\}, \{m b k l z r\},
  \{n l w j t k\}, \{n ŋ m l p t k\}, \{w n m r k t v ʃ ʒ\}.} n < m,
ɳ, ŋ < ɳ << l, ɹ < r, ʎ, ʁ < ɭ, ɽ << t < k, p < s, z, c, q, ʃ < b, d,
g, x h << w, j.  There is a slight place hierarchy: alveolar < velar <
retroflex or tap.  Classes percolate, such that in complex codas, if
Nasals, Resonants and Stops are permitted you usually expect \I{n, r,
  s, nr, ns} and \I{rs} as coda sequences.  Other orders are possible,
but the above rule is common-ish.

Hierarchy of clusters (\I{s} = sonorant, \I{o} = obstruent), word
initial: \I{os} < \I{oo} < \I{ss} < \I{so}; word final: \I{so} <
\I{oo} < \I{ss} < \I{os}.  Onset clusters tend to avoid identical
places of articulation, which leads to avoidance of things like
\textit{*tl, dl, bw,} etc., in a good number of languages. /j/ is
lightly disfavored as \I{c2} after dentals, alveolars and palatals;
/j/ and palatals are in general disfavored before front vowels.

Languages with \textit{sC-} clusters often have syllabic codas.
s+\I{stop} < s+\I{fric} / s+\I{nasal} < s+\I{lat} < s+\I{rhot} (the
fricative and nasal are trickier to order).

Even if a particular \I{c} is a permitted coda, its allowed
environment may be quite restricted. Potential constraints: forbidden
before homorganic stop; or homorganic nasal; geminates
forbidden. Solutions: delete with compensatory vowel lengthening;
debuccalize (become fricative, glottal stop, delete without
compensation; nasal deletion with nasalized vowel remaining).

Lower vowels are preferred as syllabic nuclei; high vowels are more
prone to syncope (either midword or finally).  Content words less
likely to elide.
% https://linguistics.stonybrook.edu/sites/default/files/uploads/u26/publications/QP1FinalDraft.pdf

Sonority hierarchy:
\begin{quotation}
\noindent low vowels > mid vowels (except /ə/) > high vowels (except /ɨ/) > \\
\indent /ə/ > /ɨ/ > glides > laterals > \\
\indent flaps > trills > nasals > /h/ > \\
\indent voiced fricatives > \\
\indent voiced stops and affricates, voiceless fricatives > \\
\indent voiceless fricatives, voiced stops and affricates > \\
\indent voiceless stops and affricates
\end{quotation}

Vowel devoicing starts with lowest sonority  (/i/, /u/).

Hiatus resolution: 1) elide, contract, add glide /j/ or /w/; 2)
intrude /h/ or /ʔ/; 3) intrude /t/ or /r/.

Metathesis (where P = /p, b, m/, etc.).  Obstruent + resonant can
switch.  \I{pk} > \I{kp} occasionally, but not the other way.
\I{t\{pk\}} > \I{\{pk\}t} reasonably common.

% http://linguistics.berkeley.edu/phonlab/annual_report/documents/2007/Hyman_Phono_Universals_PL.pdf
Many things can happen to a consonant following a nasal: NT > ND; NS,
NZ > NTS, NDZ; NT > NTʰ; ND > NN; but also: ND > NT; NTʰ > NT; NN >
ND.

Consonant strength hierarchy (in some African languages, generalized):
continuants, semivowels and ʔ > voiced stops > nasals > voiced and
nasal geminates, voiceless stops > voiceless geminates > nasal
clusters.

Some suffixes may occur only in pausa, or are different there.

Syllable weight: closed syllables may be heavy or light depending on
the coda consonant, with sonorants more often making weight and stops
not (in general CVV > CVC > CV).  Also, low vowels are generally
heavier than high, and central vowels less heavy than non-central. You
can get word-length dependent affix allomorphs of the same number of
syllables, but with heavier vowels in one and lighter in the other.

Consider palatalization, labialiazation, nasalization, etc., as
phenomena which apply to a \textit{syllable} rather than a particular
segment. This can generate substantially differing outcomes in
historical processes (\textit{v.} Chadic family), especially as
differing branches apply the effects more closely or more distantly. 
% https://www.eva.mpg.de/lingua/conference/08_springschool/pdf/course_materials/Wolff_Historical_Phonology.pdf
% "Issues in the historical phonology of Chadic languages"

For determining stress accent, Malagasy has some light final
syllables. These are the remnants of final consonants that got vowels
tacked on after a switch to a dominant \I{cv} structure (and the other
final consonants deleted).

\subsection{ATR Harmony}
Some systems are dominated by the feature of the word root, causing
harmony both to the left and the right. Other systems are based on
syllable position, \textit{e.g.,} a system where \I{-atr} vowels can
follow, but not precede, \I{+atr} vowels. An originally larger vowel
inventory may reduce, giving a situation where one apparent vowel can
trigger \I{+atr} in some words and \I{-atr} in others. Different
morphological systems may take different rules within one language.

May apply to all vowels, or just to \I{-atr} /ɪ ʊ/ (or /ɛ ɔ/).

/a/ may be transparent, or may block. Or may be \I{-atr}, with /ə/ the
\I{+atr} version.

Common systems: 9V /iɪeɛaɔoʊu/; 7V(M) /ieɛa(ə)ɔou/ or 7V(H)
/iɪɛa(ə)ɔʊu/. \I{-atr} usually dominant in 7V(M), \I{+atr} usually
dominant in 7V(H).

\subsection{Tone}
In many Dogon languages, postnominal adjectives and demonstratives
(but not numbers or other quantifiers other than one) cause all tones
to lower in the preceding \I{np}.  If multiple adjectives follow, all
are lowered except the last. The head noun of an internally headed
\I{rc} is also so lowered.
% http://www.dartmouth.edu/%7Emcpherson/892heath.pdf - "Tonosyntax and
% reference restriction in Dogon NPs".

\subsection{Reduplication} The original word-initial consonant of a
reduplicated root may undergo lenition, or fortition.  Malagasy
\LL{fantatra} > \LL{\uline{f}anta\uline{p}antatra}.

The \textbf{inflectional} uses inculde: number (for nouns; for verbs
marking actor number or event plurality; in noun compounds either or
both elements may reduplicate, sometimes in free variation); very
rarely to encode possession (either of 3rd person, or 1st and 2nd,
usually with partial reduplication); frequent in verbs for
frequentative, habitual or progressive aspect, but also imperfective,
inchoative and perfect(ive).

\I{red} may be used to repair a word shape that has taken some affix.

% http://citeseerx.ist.psu.edu/viewdoc/download?doi=10.1.1.54.8895&rep=rep1&type=pdf
The \textbf{derivational} uses include: ordinal numbers from cardinal;
distribution (``3 each''); valency reduction in verbs (including
antipassive), as well as reciprocals (and mutuality in other word
classes, ``face to face''); diminutive (and both endearment and
contempt); associatives (``someone with X''); similarity; sort/kind of
N; disposition (``someone prone to X, likely to X''); common in
indefinites (``who'' > ``whoever''); in verbs, a lack of control,
disorder, carelessness, pretense, attempt; incrementality
(``gradually, little by little, one by one''); spread-out or scattered
(``here and there, looking around''); non-uniformity (``zig-zag, now
and then, in several colors, hodge-podge'').

Not infrequently used to name insects and birds, and the word for
``baby.'' 

Noun to verb (``to wear an X''), adjective or adverb; verb to noun
(agent noun, action noun, instrument), adjective or adverb; adjective
to adverb.

For adjectives and other vocabulary of quality: diminutive
attentuative, augmentative, intensification.

The sense of ``collectivity'' might be used where we would expect a
single term (``broom, flight of stairs,'' etc).

Some affixes may simply require the root take \I{red}, with Makah
having several types of \I{red} and affixes selecting one.


\section{Word Classes}

\begin{center}
  \small
  \begin{tabular}{|l|l|l|l|}
    \hline
    & \LL{Reference} & \LL{Modification} & \LL{Predication} \\
    \hline
\LL{Object} & referring, argument phrase (\LL{noun}) & nominal attributive phrase &
                                                                   predicate
                                                                   nominal \\
    \hline
\LL{Property} & deadjectival nominal & attributive phrase (\LL{adjective}) & predicate adjectival\\
    \hline
\LL{Action} & complement clause & relative clause & clause (\LL{verb})\\
    \hline
  \end{tabular}
\end{center}
% http://www.unm.edu/~wcroft/Papers/Morphosyntax-ch1-aug15.pdf

\LL{Universal One}. A lexical class used in a nonprototypical
propositional act function\footnote{Reference, modification or
predication.} will be coded with at least as many morphemes as in
its prototypical function, as in \textit{bright $>$ bright-\LL{ness}} (prototypical functions are marked in bold).

\LL{Universal Two}. A lexical class used in a nonprototypical
propositional act function will also have no more grammatical
behavioral potential than in its prototypical function. For example, a
predicate nominal will not have more inflectional possibilities than a
full verb.

Individual languages will break up that 9x9 grid differently in terms
of constructions, and where individual words end up.

\section{The Noun}

Nouns are the most frequently borrowed word class.

\subsection{Possession}
``Possession'' can cover a wide range of relationships:

\begin{grammarlist}
  \item Ownership (or temporary possession)
  \item Whole-part relationship (body part, part of an object)
  \item Kinship
  \item Attribute of a person, animal or thing (``Bob's temper'')
  \item Orientation or location (``the front of the house''; useful
    when body part terms are used for location to use different
    marking)
  \item Association (``my teacher,'' but also dwellings, house to
    homeland, and personal clothing and goods)
  \item Nominalization
\end{grammarlist}

\noindent Most languages do not have the wide range of possessive uses
found in English or Greek in the same construction.  The first three
in the list above are most central.

Marking for inalienable possession is generally smaller (fewer
syllables, no classifier, simpler construction) than the marking for
alienable.

Genitive marked on R, pertensive marked on D.  Affixed possession
markers may also induce pertensive marking.  Genitive marking
frequently has other functions, pertensive rarely.

Different systems of marking may occur in two or even three groups
depending on possessor: pronoun, proper noun, kin term, common human
noun, common animate noun, common inanimate noun (with contiguous
constructions between groups).

Splits across whole-part D: external body parts, internal body parts,
genitalia, body fluids, parts of animals, parts of plants, parts of
artifacts. 

\subsection{Classifiers}

Number classifiers are most common with ``one'' and ``two,'' possibly
obligatory only with these; rarely with different classifier forms for
higher numbers.  More classifier types may be available with lower
numbers.

With stative verbs, the classic Dixon set are most likely to take
classifiers; or postural verbs.  With transitive verbs, more likely on
high-agency, high-patientness verbs (handling, cooking, killing and
the like).  With any verb class, optional classifiers may mark
salience or ``completeness'' of the classified feature.

When not used with their expected word class, classifiers may be
derivational.

\subsubsection{Gender / Noun Class}
Variable gender assignment may code size and/or shape and/or posture
(upright vs.\ horizontal).  Phonological gender assignment (initial,
final, or both, are possible determinates) typically restricted to
nonanimates.  Function of referent may change gender (water as drink
vs.\ water as part of the landscape).

Agreement: clefting may interfere or inhibit agreement; mixed class
resolution may default to the least marked form, but the behavior for
animate vs.\ inanimate may pick different forms, and coordination of
mixed animacy may simply be avoided altogether (a different
construction, or repeating the phrase with different subjects, etc.);
phonetics may inhibit some kinds of agreement.  Semantic agreement
override: attributive < predicate < relative < personal pronoun.

Different word classes may pick different genders for the least
marked, catch-all form.

It is rare but possible to have two noun class systems operating in a
single language, with different agreement rules operating on different
word classes.  In particular, ``pronominal'' agreement for pronouns
and verb agreement\footnote{Paumarí has m./f.\ for demonstratives, but
  a separate \textit{ka}/non-\textit{ka} agreement system on the
  verbs, showing up as a prefix.} and ``nominal'' agreement for
adjectives (and sometimes numbers).  Pronominal agreement in general
is focused on pronouns, is a smaller system, aligns with animacy, sex
or humanness, and have a fairly transparent semantic basis.  Nominal
agreement systems tend to be larger, focus on animacy, sex, shape and
size, and the semantics may be much less clear.  Demonstratives align
with either pattern, from language to language.  Where the two systems
have semantic overlap, the markers may be quite different, or cover
different spaces.

Hierarchy (tendency) for site of marking: pronouns > verbs >
demonstratives > adjectives > numerals. Also possible: verbs >
adjectives > pronouns.

\subsubsection{Noun Classifiers}
Similar to gender, and historically may lead to it, but classifiers
are different.  The semantics are generally clearer; some languages
may allow cooccurance of classifiers (\I{specific general}, as in
``\I{human man} boy''); single \I{n} may take different classifiers to
determine precise meaning; may be used anaphorically (esp.\ across
clauses of a single sentence), possibly with other syntactic functions
(relative clauses).  Attend to coreferentiality.

May be small (2--3) or larger (20, or more).

``Social status'' may be encoded in human classifiers (``initiate,
known,'' etc.).  Some classifiers encode inherent nature (person, bug,
tree), some function (edible, drinkable, movable, etc.).  Multiple
classifiers will usually encode one of each.

Classifier on \I{n} may be omitted once class established, or
obligatory always.

May be separate \I{q} for ``unknown class'' vs.\ ``known class but no
more.'' 

Noun classifier systems often related to number classifier systems,
but they may be different.  Number classifiers are much less likely to
be optional.  Both systems may occur in a single \I{np}.

\subsubsection{Number Classifiers}
Sometimes there is a catch-all classifier, sometimes many nouns don't
take any classifier.

May not go beyond ten, or are dropped with 10s units.

Classifiers for humans/animates may have different forms with
different numbers; with low numbers the \I{n + cl} form can be
suppletive.

Affixed classifiers more likely to code animacy, and to be required;
independent classifiers code shape, consistency, etc., and may be more
or less optional, with different speakers at different competencies.
Rarely, affixed and independent classifiers may appear together.

A single noun may take different classifiers to focus salient
characteristic.

\subsubsection{Possessive Classifiers}
Can code: shape, consistence, animacy of possessum; relationship of
possessor to possessum (as in Oceanic indirect possession); or, very
rarely, class of possessor.

There may just be a bunch of words for ``of'' which match class
(possibly highly suppletive).

\subsubsection{Verbal Classifiers} 
Many originate from noun incorporation, possibly competing with it.
However, some verbal classifiers may be clearly related to numeral or
other classifiers in the language.  Another origin is from verbs, via
grammaticalized serial verb constructions.

Existential verbs may distinguish animate from inanimate; a few
languages elaborate existentials quite a lot (in a container, movable
vs.\ immobile, non-human animal, etc.).

\subsection{Case}
\I{sov} languages more likely to have case marking than \I{svo}.

\subsubsection{Allative} A huge range of possibilities here.
Occurring more often: true allative (``to, towards, reaching for''),
purpose (``use it for that, in order to''), conceptual (``think about,
occur to''), recipient (dative), timepoint (``at \I{time}''),
addressee (``talk to me''), perceptual (``look at, listen to''),
reason (``because of, ran from fear'').  Some less frequent
possibilities include: temporal boundary (``by/until \I{time}''),
benefactive, possessive, porportion or rate (``3 out of 4, 3 at a
time''), equivalence (``as, in exchange for''), subordinator
(``although, when, while''), emotional target (``hard for, be angry
at'').
% RiceKabata2007.pdf


\section{The Pronoun}
There may be separate forms for ``\I{prn} alone'' and/or ``\I{prn}
also'' unrelated to similar expressions for nouns.

Languages with 1/2 systems are not common, but not rare either.

Sg/du/paucal/pl is far more common that sg/du/trial/pl.

It's not especially common for sg and pl forms to be related (as in
Chinese).

Many languages:

\begin{center}
\begin{tabular}{lll}
  & Singular & Plural \\
\I{1st} & 1 & 12, 13 (and 11) \\
\I{2nd} & 2 & 22, 23 \\
\I{3rd} & 3 & 33
\end{tabular}
\end{center}

However, one might get something like:

\begin{center}
\begin{tabular}{ll}
1 & 13 (and 11) \\
2 & 12, 22, 23     
\end{tabular}
\end{center}

The basic 1/2/3 sg/pl system may be extended to include a special 12
``me and you'' form.  This may involve the innovation of a dual
throughout the pronoun system.  Or it may be just a normal system with
an inclusive/exclusive distinction.  Or a minimal/augmented system, 

\begin{center}
\begin{tabular}{ll}
Minimal & Non-minimal \\
1  & 1 + others \\
1+2 & 1+2 + others \\
2 & 2 + others \\
3 & 3 + others 
\end{tabular}
\end{center}

\noindent The non-minimal may include ``unit augmented'' (one other
person, producing a dual in many forms) and just ``augmented'' (the
``plural'').  Minimal/augmented are common in Australia, Austronesian,
South America, and a few in North America, though the unit
augmented/augmented distinction is nearly restricted to Australia.

3sg = 3pl is a common neutralization.  2pl = 3pl (and 2du = 3du) can
happen.

Using 2du, 2paucal or 2pl as a mark of respect occurs in Europe,
Oceania, Australia.

There is a relationship between the indefinite ``someone'' and
1non-sg forms (French; Caddo, etc).  Usually it's 1pl.inc that does
this. 

Question-based indefinites may be marked with some other morpheme,
often related to words for ``be,'' ``want,'' ``perhaps,'' ``or'' or
``also.''


\subsection{Demonstratives and Deixis}
% https://research.jcu.edu.au/lcrc/storeroom/sashas-folder/demonstratives-directionals-summary ``Demonstratives and directionals: summing up''

\begin{grammarlist}
  \item two way
  \item three way: proximal, distal, remote; proximal, medium, further
    away or imprecise; proximal, medial (not far, known to 1 and 2),
    distal
  \item four way: proximal (to 1), proximal (to 2), medial, distant
  \item five way: priximal visible, proximal audible, medial, distant,
    imperceptible
  \item six way (Godoberi): close to 1, close to 2, that at some
    distance from 1, that at some distance from 2, that down there,
    that (aforementioned)
\end{grammarlist}

Rarely, pronominal uses may require nominalizer.

Some systems have forms specifically used for anaphora, or only permit
particular forms to be used anaphorically (slight preference for
distal forms for this?).

In a very few languages, there may be special forms always accompanied
by physical gestures (English ``yei high'').

May have overtones of familiarity, endearment, pejorative,
empathy.  Like diminutives, only context may make clear a positive
vs.\ negative interpretation.  Even proximal deixis can be
pejorative. 

Demonstratives can be recruited as filler, both within a sentence (a
particular form will be preferred), or as ``um, ah.''

Splits: more distinctions for modifier, with fewer for independent,
pronoun-like forms. 

Adverbial locatives (``here, there'') often match demonstrative
distinctions (not infrequently by derivation or clear relation), but
may have fewer distinctions.  Rarely, particular clause positions may
have fewer options (Jarawara, one way distinction clause initial,
``here/there,'' two way distnction clause final).

Manner adverbials (``thus, like this, like that''), again often
related to pronouns, may make fewer distinctions.

Verbal demonstratives: ``do/be (like) this/that.'' Can be both
anaphoric and cataphoric.

\rara{Demonstrative agreement with addressee (in addition to usual),
  with traces of such agreement for \I{adj}.}
% https://orientalberber.wordpress.com/2012/05/29/siwi-addressee-agreement-and-addressing-aljazeera/



\section{The Adjective}
Comparatives neither rare nor universal.  Can be marked with (1) affix
and adpositional phrase; (2) adposition only; (3) coordination (``X is
big but Y is small'' = ``X is bigger than Y'').

Breaks along word class distinctions follow the adjective class: full
\I{nav} (all of noun, adjective and verb in the lexicon); \I{n[av]}
(i.e., property-concept words go with verbs); \I{[na]v}; and \I{[nav]}
(all three classes conflated).
% http://www.ualberta.ca/~dbeck/StJuste.pdf

Resultative adjectives are most likely in languages in which there are
complex verbal predicates (serial verbs and particle verbs). ``I
hammered it flat'' slots into complex verbal constructions naturally.
French, for example, simply doesn't have these.
% https://www.uni-salzburg.at/fileadmin/multimedia/Linguistik/documents/On_predicting_resultative_adjective_constructions-June-2016.pdf


\section{Numerals}
Bases found in human languages: 2, 3, 4, 5 (hand), 6, 8 (like 4,
counting the spaces between digits), 12, 20, 60.  Hybrids: 5+20(+80),
2+5, 10+20.  Base 10 is by far the most common.  Lower bases are less
common, and will generally only go up to a few powers, with a base 4
system, for example, only reaching 8, 16, or 32.

The higher power units may not be usable alone. That is, unlike
English ``ten,'' some languages may require it be ``one ten'' in the
same way 20 is ``two tens.''

The words for 2 and 2nd often convey sense of ``another''
(``secondary''). 

Overcounting: some systems overshoot, then work backwards, so that
``seven thirty'' is 27 (old Turkic).  Subtraction: 8 = two from ten;
31 = one plus 10 from 2*20.

A very few languages (PNG) uses different bases for counting different
things (as in Bukiyip, which uses 3 and 4).

% http://sajms.com/wp-content/uploads/2016/03/A-Typology-of-Rare-features-in-Numerals.pdf ``A typology of rare features in numerals''

\section{The Verb}
``Labile transitivity'' is very common.  Be clear on \I{s = a}
\textit{vs.} \I{s = o} for intransitive constructions.  Some languages
may skew in favor of one or the other, but others do not.  English
``cook,'' with transitive and both intransitives (``the chef cooks;
the chef cooks stew; the stew is cooking'') is unusual.

Lability is a dodge!  If the language has rich mechanisms for changing
transitivity, it's less frequent.

Labile verbs tend to cluster semantically, 1) destruction and strong
property change, ``break, boil, freeze, dry, go out, melt, dissolve,
burn, destroy, break, split, kill/die;'' 2) motion and spatial
configuration, ``rock, roll, sink, spread, close, open, connect,
rise/raise, stop, fill, turn;'' 3) phase, ``begin, finish;'' 4)
non-physical effect, ``change, improve, develop;'' and 5) verbs with
an animate patient, ``wake up, learn, gather.''  A single language may
pick several clusters, or only a few items from one.
% Letuchiy, A., "Interpreting the spontaneity scale."

A transitive verb may be used intransitively in an extended
intransitive sense, too.

Verb pairs (``die/kill,'' ``eat/feed'') for which the S/O role are
primarily animate are more likely to pattern with one basic
transitivity for the simplest stem (i.e., ``kill'' is primary with a
derivation for ``die''), while those with primarily inanimate S/O role
verbs (``boil/boil,'' ``burn/burn'', ``fall/drop'') will also have a
primary transitivity (in English, these are often labile or
suppletive).  In general, though, the basic role for animate verbs
will be intransitive, with augmentation for the transitive form.  On
the other hand, in the large, some few languages are fairly
intransitive, some quite strongly transitive.

Stative verbs often mark fewer tense or aspect distinctions (such as
Eng.\ -ing, ``I am knowing''). Or, they may take some additional
marking obligatorily (Turkish).

Among other categories, some languages have verbs that mark ease or
difficulty.  The marking for ``ease'' on a transitive verb may signal
a small \I{o}.

In richly marked verbs, these may not be marked: \I{3rd} inanimate
sobjects, all \I{3rd} objects, \I{3rd} topical subjects, \I{3rd}
absolutives, all \I{3rd} of any kind.


\subsection{Affix Order}
The fewer person markers there are, the more likely they are to be
prefixing.  But: object marking on the verb is prefixing with more
than chance frequency.  Prefixing in general is less common with
\I{ov} than with \I{vo}, except for object marking, with a distinct
preference for object prefixing with \I{ov}.
%http://www.zas.gwz-berlin.de/fileadmin/material/40-60-puzzles-for-krifka/pdf/cysouw.pdf

If \I{t} and \I{a} are on the same side of the verb, \I{a} is closer;
if different, the order is \I{t Verb a}.

% Mine this paper for additional data. http://linguistics.buffalo.edu/people/faculty/dryer/dryer/KeenanDryerPassive.pdf
\subsection{Passive} Most often synthetic, with the marking usually,
but not always, closer to the verb stem than \I{tam} marking.

If auxiliary, most common intransitives are \textit{be, become,} and
verbs of motion; most common transitives are \textit{get, receive,
  suffer, touch;} even \textit{eat} is attested.

Accessibility in promotion of passive: \I{direct object > indirect
  object > adjunct}.

Actor most often represented by an oblique: instrumental, locative
(preposition, \textit{by}, ὑπο, etc.), genitive.  Rarely, a special
marking is employed.

\subsection{Valency}
Possible that applicatives can only be used for animate arguments.
Less often, may only be used with intransitive verbs.

It may be that applicatives in general are less frequent in languages
with rich case systems (less necessary).

Antipassive marker may code for humanness (Rgyalrong).


\subsection{Semantic Types and Their Frames}

\begin{center}
\begin{tabular}{lllll}
\I{affect} (\E{hit, cut}) & Agent & Target & Instrument \\
\I{giving} (\E{give, lend}) & Donor & Gift & Recipient \\
\I{speaking} (\E{speak, tell}) & Speaker & Addressee & Message & Medium \\
\I{thinking} (\E{consider}) & Cogitator & Thought \\
\I{attention} (\E{see, hear}) & Perceiver & Impression \\
\I{liking} (\E{like, love, hate}) & Experiencer & Stimulus
\end{tabular}
\end{center}

In some languages, an \E{affect} verb may require an inanimate agent,
usually specific, such as food making a person sick, \textit{etc}.

In very few languages are \I{liking} verbs like \I{annoying} verbs,
where the experiencer and stimulus are in \I{o} and \I{a} roles.
However, extended intransitives, with an oblique experiencer, are a
bit more common.

\I{attention} and \I{liking} may have extended intransitive frames.
Or dative subjects.  Or oblique objects.

\I{want} verbs may have extended intransitive frames.

The types with more than two arguments may have lexical or
construction splits to determine which of the arguments is in the
primary \I{o} slot (``tell'' vs. ``speak (French)'' vs. ``say'').

For \I{give} there may be i) single verb with different constructions,
ii) gift role is always in \I{o} function, iii) recipient role always
in \I{o} function (rare; amusingly, with \I{dat} for the gift role).
Or, ii and iii may have different lexical items.


\subsection{Associated Motion}
In addition to the basic trans- and cislocative, which link a motion
to the main event of the verb, they may code relative time of motion
(prior: \I{do.arriving}, ``when, after going, X''; simultaneous:
\I{do.going}, ``while going, X''; prior: \I{do.and.leave}, ``before
going, X''). They may distinguish simple deictics (``come in order
to'') vs.\ associated motion (``come while doing'').  There may be a
distinction between ``go'' and ``come'' vs.\ ``go back'' and ``come
back.''  Arrernte distinguishes a ``quickly'' or hurried series, as
well as ``do coming through.''  No deictic center may be present for a
small subset (e.g., ``do here and there'') of such forms.  In some
systems, the elaboration for \I{s/a} is reduced, but there is a series
for \I{o} on transitive verbs,\footnote{With deixis oriented to the
  verb subject.  Not sure if that's universal.} ``I saw-\I{go(o)} a
man'' for ``I saw a man going away from me.''  Finally, the target of
motion may be coded for stability (temporary or permanent,
``enter.\I{come.temporary}'' ``come in,'' vs.\ ``sit.\I{go.perm}''
``go and sit and stay there'').

These locate an event in space much like tense locates an event in
time.  Though using these may add locative arguments to clause, the
main verb is always foreground.  Further, they may link discourse,
with things like ``he came, and he argued-\I{cis} with me,'' with
``come'' and \I{cis} marking the same path information.

Directional affixes may be grammaticalized in various ways (inward =
perfective, downward = progressive, upward = imperative; cis =
inceptive, change of state, trans = endpoint of activity); or more
metaphorical senses (down = not fully satisfactory, up = better, back
= return to health or satisfactory state).

Like tense and aspect, these locatives may be restricted or forbidden
in subordinate or nominalized clauses (these are inflectional, not
derivational). 
% https://ecommons.cornell.edu/bitstream/handle/1813/13037/Deal.pdf?sequence=2

Associated motion appears to occur in about 1/3rd of languages, with
certain hotspots (Australia, the Andes and Amazonia).  It is unlikely
in languages with serial verbs.

% http://linguistics.berkeley.edu/~fforum/handouts/vuillermet_AM_fforum_handoutFINAL.pdf
% http://www.academia.edu/7866301/Reconstructing_the_category_of_associated_motion_in_Tacanan_languages_Amazonian_Bolivia_and_Peru_
% https://escholarship.org/uc/item/85r367r3#page-40
% http://www.ddl.ish-lyon.cnrs.fr/trajectoire/23us23efd5ps/Presentations/Pres_Guillaume091107.pdf

\subsection{Participles}
% Ksenia Shagal, ``Towards a typology of Participles,''
% https://helda.helsinki.fi/bitstream/handle/10138/177418/Towardsa.pdf
Participles may have an \textit{inherent orientation} or a
\textit{contextual orientation.}  European languages have inherent
orientation, where the role participant is determined by the
participle form (active participle for agent, passive for patient).
In contextually oriented participles, several different participants
(agent, patient, location) can all be encoded, and usage determines
the interpretation.

Participle forms often have several functions in addition to
adjective-like attribution: adverbial (``watching the children''),
clausal (``I saw he was not running'').  Participles and
nominalizations may be hard to distinguish formally.

Even languages without an adjective class can have things quite like
participles, which are used in relative/attributive constructions. 

\subsection{Converbs}

Moderately more likely in SOV languages. Most things called converbs
require same subject, but an argument can be made for inflected
converb-like functions being part of the same phemomenon (cf.\
Coptic).  Forms not inflected for person may or may not require same
subject, though same subject is most common.  Argument marking may be
identical to a main clause, or different.  There may be restrictions
on negation.  Rarely, converb clause may occur at the other margin of
the clause, often with iconic restrictions, as in a purpose or
intention converb allowed to follow the clause, while imperfective may
not (in an SOV language).

Simplest functions: aspectual, imperfective (simultaneous) and
perfective (anterior, usually). Some languages have many forms (all
from Norther Akhvakh): locative (may take location case marking),
inceptive (``from the moment X-ing began''), immediate (``as soon
as''), anterior (``before X-ing''), imminent (``just before X-ing''),
non-posterior (so fast that the converb event may not have time to
occur at all, ``come down here-\I{cvb} before something bad
happens''), conditional, concessive (``although''), similative (``in
the same way as''), gradual (``the more..., the more...,'' with the
main clause just normally marked), cause (``because''), purposive
(``in order to'').

In languages with rich converb inventories, some may be highly
restricted with respect to the main clause verb: Akhvakh progressive
converb only occurs with ``be, remain, see, find.''

In Turkic languages (and some others), converb forms are the \I{lex}
in auxiliary constructions.

% http://www.deniscreissels.fr/public/Creissels-adv.sub.Akhv.pdf
% http://pj.ninjal.ac.jp/vvsympo/NINJAL2013_Shluinsky_handout_changes.pdf
% http://www.livingtongues.org/docs/AVCsOT1.pdf

\section{The Adverb}
In languages otherwise without a well-defined adverb class, or a
regular way of forming adverbs, there may be root adverbs for the
domains \I{speed} (``quickly, slowly'') and \I{value} (``well,
poorly, bad''). In addition to root adverbs, these can show up as
derivations from adjectives which may differ from other adverb-like
constructions somehow (such as being zero-marked). These two semantic
cores are rarely the only one in any particular language.
%http://www.ling.su.se/polopoly_fs/1.99116.1346331417!/menu/standard/file/Hallonsten_Halling_Pernilla.pdf


\section{Indefinites}
X-linguistically, indefinites derived from interrogatives are about
twice as common as those derived from nouns. If a derivational element
is used on interrogative indefinites, often related to ``be, want,
perhaps, or,'' or ``also.'' A few linguages split function, with
interrogatives for ``any'' indefinites and nouns for ``some''
indefinites.

Indefinites off the the left of the Haspelmath map are more likely to
be appreciative, and those to the right depreciative.
% http://www.philol.msu.ru/~otipl/new/fdsl/abstracts/bylinina.pdf


\section{Number}

The distinction between ``paucal'' and ``plural'' is context
dependent. 

In a small number of languages the unmarked noun is collective, with
separate marking for singulative and plural.  Or, a few nouns might be
like that (\textit{cf.} Welsh).

If there is a trial, its use may signify salience of the number, with
the plural being used for three much of the time.

Number suppletion in verbs: not common, aligns ergatively.  Usually
only 1--4 verbs take it, but may reach a couple dozen.  If there is a
sg/du/pl distinction, at least some verbs will just have sg/pl (or
non-pl/pl).  Among intransitives, the posture verbs are most commonly
suppletive, \textit{sit, stand, lie,} and most likely to show numerous
number distinctions (du).  Next: \textit{enter, go, be big, die/be
  dead, hang, arrive, run, come, fall, cry, be little}.  Transitives:
\textit{kill, put/place, throw, give, break, take, bring, carry}.


\section{Constructions}

Any schematic construction may, like lexical constructions, be
polysemous.  For example, the English ditransitive:

\begin{grammarlist}
  \item \I{X causes Y to receive Z} (central sense), ``Joe gave Sally
    the ball.''
  \item Conditions of satisfaction imply \I{X causes Y to receive Z},
    ``Joe promised Bob a car.''
  \item \I{X causes Y not to receive Z}, ``Joe refused Bob a cookie.''
  \item \I{X acts to cause Y to receive Z} at some future time, ``Joe
    bequeathed Bob a fortune.''
  \item \I{X enables Y to receive Z}, ``Joe permitted Chris an
    apple.''
  \item \I{X intends to cause Y to receive Z}, ``Joe baked Bob a cake.''
\end{grammarlist}

Similarly, a particular language may have one construction for a
particular job, such as a ditransitive construction, but others have
several with different pragmatic or pivot significance (``I gave him
the book'' \textit{vs.} ``I gave the book to him'').

\subsection{Alignment}
The non-ergative clauses in so-called split ergative systems are
rarely nom-acc (\I{a = nom, p = acc, s = nom}).  Instead one can get
\I{a = abs, p = abs, s = abs} (common), \I{a = abs, p = obl, s = abs}
(common) or even \I{a = erg, p = abs, s = erg} (common in Mayan).  The
aspect splits often reflect (historically) intransitive \I{aux}
constructions.  Thus, in rare circumstances, one may get splits in
future clauses, or with negation, if they reflected an original
\I{aux} construction.

\subsection{Complement Clauses}
These come in three types.

1.  Fact type, indicate that something did take place.  Similarly
marked to main clause.  If subject is the same across clauses, it
isn't likely to be omitted.  Usually marked with some complementizer
element which will have other functions in the language (often ``say''
or ``be like'').  Complementizer may code reliability (sure fact
vs. possible fact).

2. Activity type, indicating extension in time.  Often similar to a
noun phrase, but will still have a subject.  If the subject is the
same, it may be omitted.  Or the verb may have a special form.
Generally less specified in TAM and negation than a main clause; may
not include same bound pronominal elements.

3. Potential type, typically less like a main clause than the Fact
type and less similarity to a NP than the Activity type.  In some
languages, the subjects must be the same.  Reduced TAM and pronominal
marking.  Implicit type reference to same or posterior time.
Generally a special verb form (``infinitive'') or may take marking
similar to dative or some other case.

Languages may range from one to 5--7 complement types, with subtypes.

Attention verbs (``see, hear, show'') typically take Activity
complement.  May take Fact, for completed actions or of state.
``Find, discover'' are expected to take Fact.

Thinking verbs (``think (of, about, over), consider, imagine'') take
Fact, or sometimes Activity (``think about'').  ``Assume, suppose''
take Fact.  ``Remember, forget'' take Fact, with English unusual in
taking potential (``I remembered to shut off the stove'').  ``Know,
understand'' take Fact or Poetential.

Deciding (``decide, resolve, choose'') take Fact or Potential.

Liking (``love, prefer, regret, fear'') take Activity or sometimes
Fact.  ``Enjoy'' takes Activity.

Speaking has several subtypes.  ``Say, inform, tell'' usually take
Fact.  ``Report'' takes Fact or Activity.  ``Describe, refer to''
takes Activity.  ``Promise, threaten'' takes Potential, which may be
in the indirect object slot.  ``Order, persuade'' generally take
Potential. 

\subsection{Desententialization}

The most desententialized is purpose clauses: purpose $<$ before $<$
after, when $<$ reason, reality condition.  Also: phrasal, modal $<$
desiderative, manipulation\footnote{\textit{urge, suggest,} etc.} $<$
perception $<$ propositional attitude,\footnote{\textit{believe} and
the like} knowledge $<$ utterance.

% Add more from: http://www.academia.edu/2099252/Argument_coding_and_clause_linkage_in_Australian_Aboriginal_languages


\subsection{Negation}
In negation, some verbs may take different object marking.

Consider a separate prohibitive.

Circumfix all over planet, just not common.

Possible distinctions, beyond simple ``not:'' noun vs.\ non-noun, verb
vs.\ non-verb (where adj.\ may pattern with either); different negator
for the future; prohibitive; participle negation; negative particle
for non-existence (which may be like ``no'' or plain negation for
nouns).  Non-verb negator can be grammaticallized from ``other,
different; refuse, not want.''


\subsection{Conditionals}

The protasis of a condition and formal marking of the topic may be
identically or similarly marked (including ``if X \I{adp}'' or the
like for topic marking).


\section{Discourse}

\subsection{Topicality}
Generic topicality hierarchies: speaker > hearer > \I{3rd}; human >
animate > inanimate; agent > dative > patient; large > small (and
adult > child); possessor > possessed; definite > indefinite; pronoun
> full \I{np}.

Referential hierarchies: Speech-act participants > Kinship/Name >
Human > Animate > Inanimate; Specific > Non-specific referential >
Generic; Known/Topical/Thematic/Definite > New.  Different languages
take different approaches for SAPs, with both 1 > 2 and 2 > 1 found
(the latter a politeness matter, apparently) and number may play a
role, too. Found in the wild: \I{1pl/2pl > 1sg > 2sg}, \I{1pl > 2},
\I{2pl > 1 > 2sg}.
% http://elanguage.net/journals/lsameeting/article/viewFile/2826/pdf

\section{Lexicon}
Consider a few phonesthemes.

Diminutive/medial/augmentative may code gender of the speaker (Weining
Ahmao, in the classifier system).

Special formal/elevated vocabulary word shapes may have several
patterns of systematic relationship between it and base level words
(Javanese).

% http://www.linguistics.ucla.edu/faciliti/wpl/issues/wpl17/papers/40_polinsky.pdf
\I{vso} languages are more likely to have even noun-to-verb ratios
(lots of verbs derived from nouns), while \I{sov} languages are more
likely to have \I{n+v} idioms taking up the slack, resulting in more
nouns.  This includes \I{svo} languages that lean \I{ov}.  Regardless,
there are almost always more nouns than verbs (though perhaps not by
many).

Respect forms may contain morph for ``lord'' or ``sky.''

Some Austronesian languages have an ``anger'' vocabulary, not
swearing, for verbs and nouns.  Some common deformation patters.

Hunting, fishing or territorial languages: to conceal your intent from
animals or spirits, or with tabu considerations; circumlocution and
non-systematic deformation common practices for this.

Body parts become special concepts in stages: 1) a region of the human
body, 2) a region of an inanimate object, 3) a region in contact with
an object and 4) a region detached from the object.  The landmark path
(``extremity, peak'' $>$ ``head'') goes in reverse.
% http://www.sciencedirect.com/science/article/pii/S0388000112000423

Over time words move from external and objective to subjective and
grammaticalized (``boor'' from farmer to oaf, ``feel'' from touch to
experience emotion, ``insist'' from perservere to demand to believe
strongly).
% https://www.reddit.com/r/linguistics/comments/34r1vk/where_can_i_find_a_list_of_metaphors_that_have/

Once a conceptual metaphor has taken root (\I{seeing} is
\I{understanding}) vocabulary merely related to the concept of seeing
may also be dragged into the metaphor later.  You don't expect
``brilliant = intelligent'' to happen until the first metaphor is well
established.
% http://www.e-revistes.uji.es/index.php/clr/article/viewFile/1363/1206

\subsection{Temperature}
Core words for temperature may make only two distictions (``hot/warm''
vs.\ ``cold/cool''), three (``cold/cool'' vs.\ ``warm'' vs.\ ``hot''),
or four (``cold, cool, warm, hot''). Some languages may have unique
intensifiers for extreme temperatures, usually related to words for
processes or things that exemplify the temperature (such as ice or
burning).

Temperature may refer to three domains: tactile (``the plate is
hot''), ambient (``it's hot here''), and personal-feeling (``I am
hot''). Languages like English or Italian may use a single word
(``cold,'' ``freddo'') for all three; there may a unique word for each
domain, as in East Armenian; or the domains may be split up among two
words.  Expressions for the personal-feeling domain are most likely to
be different in some way.  The ambient domain generally makes the most
distinctions, the personal-feeling the fewest.  Personal-feeling terms
may be restricted to ``uncomfortably hot'' and ``uncomfortably cold.''

Terms in the ambient domain may mark source of the heat, humidity, the
effect on breathing, windiness, etc., such as one for ``hot'' and
another for ``hot and humid.''  Some terms may be restricted to
season, such as Palula ``cool'' \textit{šidaloó}, which can refer to
pleasant coolness in summer, but is not used to describe not-too-cold
in the winter.

There may be separate tactile domain temperature terms for food and
especially water.

\subsubsection{Metaphors and Idiom}
Common source: \I{bodily manifestation for emotion}. The face often
(actions, colors), but also eyes. Other organs get up to shenanigans,
including blood.

Colors seem to often carry emotional connotations.  What a
particular color signifies will vary from culture to culture: in Thai,
a green eye or face signify anger.

Metonymy: result for whole, salient feature for whole (where whole
might be thing or event), instrument for action, effect for cause,
producer for product; both whole for part and part for whole.


\section{καὶ τὰ λοιπά}

In Kolyma Yukaghir there is a separate \I{acc} just for use on \I{1/2
sg/pl} when the subject is \I{1/2 sg/pl}, the ``pronominal
accusative.'' 

Form dependency: each element's forms may depend on choices made in
the feature above them (such as fewer tense forms in the negative, for
example).  The final three are potentially interdependent: polarity
$<$ tense, aspect, evidentiality $<$ person, reference classification
(gender, etc.) $\Leftrightarrow$ number $\Leftrightarrow$ case.

Hierarchy of diminutives and augmentatives: noun $<$ adj., verb $<$
adv., numeral, pronoun, interjection $<$ determiner.  These markers
may be distributed over several words in a sentence.

If the noun takes prefixing morphology, so will verbs.

Serial verb constructions more likely in \I{svo} languages than in
\I{sov}.  \I{sov} languages are more likely to use converbs, though a
very few have both converbs and SVCs.  Having a generalized ``and''
(\textit{i.e.,} the same for \I{np} and \I{vp}) is weakly correlated
with not having SVCs.

Pseudo-coordination (``try \uline{and} see'') is typically restricted
to a very small set of verbs, and in Germanic languages at least is
often used in aspectual constructions (``sit and ...'' in Norwegian
for progressive, ``return and...'' in Arabic for repetition).
Straight-up aspect verbs (``begin'') can also use such a
construction. ``Go'' and ``come'' with purpose may fall into this
construction.  Also used for: causative (``make/cause and''),
intention (``plan and''), reason, conditional.
% http://publish.illinois.edu/djross3/files/2014/05/Ross2014_Lisbon.pdf

Writing systems: vertical and horizontal lines, easiest to identify,
most frequent component of most symbols in a set. Obliques and
diagonals do not mix with vertical and horizontals too much (K, A, Z,
less frequent than E, H, F or W and X). Vertical symmetry (M, A, W)
more common that horizontal (K, D, E).
% http://www.shh.mpg.de/654434/legibility-emerges-spontaneously-rather-than-evolving-over-time

\subsection{Oblique Strategies}
Don't ask what it does, but how it moves.

Don't ask what it does, but what it looks like.

Don't ask what it does, but who it is important to.

Don't ask what it does, but who it hangs out with.

Verb: action vs.\ means vs.\ result.

\section{Sound Changes}
{\small
\input changes
}

\section{Sound Systems}

\subsection{Chipaya} Onset clusters: /s ʃ/ + /p/ + (/x/); /s ʃ/ + /k q/ +
(/x xʷ χ χʷ/); /t/ + /x xʷ χ χʷ/; /tʃ l/ + /x/.  Possible codas: /x χ/
+ /p t k q l r/ + (/t/); /xʷ χʷ/ + /k q/ + (/t/); C + /t/. 

\subsection{Assiniboine} Onset clusters: /p/ + /t s ʃ tʃ/; /tk/; /k/ +
/t s ʃ tʃ m n/; /s ʃ/ + /p t k tʃ m n/; /x/ + /p t tʃ m n/; /mn/.  No
codas. 
\end{document}
